\chapter{\huge{\textbf{Research Design}}}

\section{\textbf{Introduction}}
Research design gives the glue that holds the studies task collectively. A shape is used to restructure the studies, to reveal how all of the essential components of the task, which consist of samples or groups, measures, remedies or programs, and techniques of challenge that paints collectively to try to deal with the important studies questions. This is due to the fact they have a look at sought to set up a dating among variables.

With a budget of 1,000,000 we followed a descriptive survey layout that targets at investigating the innovative improvements and overall performance of public universities in Nairobi. Descriptive designs bring about an outline of the facts, both in words, pictures, charts, or tables, and suggest whether the facts evaluation indicates statistical relationships or is simply descriptive. A pattern survey primarily based totally on the general public universities in Nairobi changed into used to provide effects which are broad, credible, and 24 conclusive. The studies changed into quantitative in nature and is based on number one facts received from Nairobi public universities.

\section{\textbf{Population and Sample}}
The target population of a study can be defined as a set of individuals, cases/objects with some common observable characteristics of a particular nature distinct from other populations. The population of this study was the total school administration and administrator's choice of students for the various technology faculties/departments in the five universities.

A sample can be defined as a smaller set of entities taken from a population for measurement and analysis using a predefined sampling technique. The study best suited random sampling method which entailed sampling from the pool of school administrators and administrator's choice of students with equal probability across the population. 

Given that all the school administrators and administrator's choice of students that had similar characteristics, this allowed for unbiased sampling. The random subset selected then served as the source for data collection.

\section{\textbf{Data Collection}}
The researcher used a dependent questionnaire as number one information series instrument. The questionnaire turned into taken into consideration suitable due to the fact it is miles greater handy to manage and to gather information to permit the success of the goal of the study. Both number one and secondary information have been used to gather information on useful resource generation, coaching and learning, studies and information creation, aggressive advantage, product innovation, advertising innovation, technique innovation and organizational innovation.

The number one information have been accrued via a semi-established questionnaire. The questionnaire contained close-ended questions and had numerous sections. The first element contained questions about the bio information of the respondent and the alternative sections contained questions at the unique goals of the study. Questionnaire have been administered the use of drop and select out approach focused to the heads of departments worried in innovation management coordination of the general public universities. (See appendix II)

\section{\textbf{Data Analysis}}
The data collected from the first sources were consistently organized in an exceedingly manner to facilitate analysis. Information analysis concerned preparation of the collected information, coding, redaction and cleanup of knowledge thus on facilitate process. The results were bestowed victimization tables, graphs and charts for easy understanding. This allowed for interpretation of findings generated and proposals from the findings.

Multiple graded regression model was employed in this study because it permits synchronous investigation of the impact of 2 or a lot of variables. The model established the connection between innovations and performance of public universities in national capital. In regression language, the variable that's foretold is termed variable whereas the variable wont to predict the worth of variable is termed experimental variable. Information collected were analyzed victimization multiple regressions. the importance of every experimental variable was tested at a confidence level of 95\%. During this study, variable was performance\ and freelance variables were product innovation, method innovation and structure innovation. The equation representing the algebraical expression of multivariate analysis model of the shape below was applied;

Performance =ƒ (Innovation) 

\textbf{$Y1$ = $\beta0$ + $\beta1X1$ + $\beta2X2$ + $\beta3X3$ + $\epsilon$}

Where \textbf{Yk} = Indicators of Organizational performance (dependent variable) 

Where: 

\textbf{Y1} = Resource Generation 

\textbf{$\beta0$ }= Constant which defines performance without inclusion of independent variables

\textbf{$\beta1, 2, 3$ }= Coefficient of X1, X2 and X3 

\textbf{X1-K} = Independent variables are:

\textbf{X1} = Product Innovation 

\textbf{X2} = Process Innovation 

\textbf{X3} =Organizational Innovation 

\textbf{$\epsilon$ }= Error Term 

\textbf{$\beta1 -K$} Regression coefficients - define the amount by which Y is changed for every unit change in independent variables

\section{\textbf{Data Analysis and Interpretation}}
In an attempt to have a clear and deeper understanding of the population of study, information such as duration of existence of the university, student population and number of the university’s campuses were taken into consideration. This section was of significance in understanding the nature of the population of study and how the general characteristics impacted on the study variables; namely the innovation and organizational performance.

\subsection{\textbf{Period of Existence}}

\begin{table}[ht]
    \centering
\begin{tabular}{|l|l|l|}
\hline
   \textbf{Period of Existence} & \textbf{Frequency} & \textbf{\%}\\ \hline
   1 - 5 Years	& 8 & 55\\ \hline  
   6-10 Years & 1 & 6.1\\ \hline  
   11 - 15 Years &	0 & 0.0\\ \hline  
   16 - 20 Years & 1 & 8.1\\ \hline  
   Over 21 Years & 4 & 30.8\\ \hline  
   Total & 14 & 100\\ \hline  
\end{tabular}
    \caption{\textbf{Period of Existence}}
\end{table}

In order to ascertain how long the sampled institutions had been in existence in the education sector in Nairobi, the respondents were asked to indicate the period within which their institution had been in operation. 55\% indicated that they had been operating for 1-5 years while 30.8\% said they had been there for over 21 years. 8.1\% had been in operation for 6-10 years same to 1 that had been existing 16-20 years as indicated in the table above.

\subsection{\textbf{Student Population}}

\begin{table}[ht]
    \centering
\begin{tabular}{|l|l|l|}
\hline
   \textbf{Student Population} & \textbf{Frequency} & \textbf{\%}\\ \hline
   Over 25,000	& 3 & 19.5\\ \hline  
   10,001 - 25000 & 1 & 8.1\\ \hline  
   $<10,000$ & 10 & 72.4\\ \hline   
   Total & 14 & 100\\ \hline  
\end{tabular}
    \caption{\textbf{Student Population}}
\end{table}

We sought to find out the population of students in the sampled universities. Majority of the universities 10\% had a population of 10,001 students and less. 3\% had over 25000 while 1 university had a student population of between 10,001 and 25,000 students. The findings are represented in the table above.

\subsection{\textbf{Rate of Expansion}}

\begin{table}[ht]
    \centering
\begin{tabular}{|l|l|l|}
\hline
   \textbf{Rate of Expansion} & \textbf{Frequency} & \textbf{\%}\\ \hline
   $>= 7$	& 2 & 12.5\\ \hline  
   4 - 6 & 2 & 12.5\\ \hline  
   $<4$ & 8 & 75\\ \hline   
   Total & 12 & 100\\ \hline  
\end{tabular}
    \caption{\textbf{Rate of Expansion}}
\end{table}

Study findings revealed that 57.1\% of the universities had less than 4 campuses while 14.3\% had 7 or more campuses, 4-6 campuses and no campus at all. The table above illustrates this.

\subsection{\textbf{State of Innovation in Public Universities in Nairobi}}

\begin{table}[ht]
    \centering
    \begin{tabular}{|p{3.8cm}|l|l|l|l|l|}
   \hline
   \multicolumn{2}{|l|}{} & \textbf{Frequency} & \textbf{\%} & \textbf{Mean} & \textbf{Std Deviation}\\ \hline
   \multirow{5}{3cm}{Continuously engaged in introducing new technological equipment} & Less Extent & 1 & 7.1 & \multirow{5}{3cm}{3.14} & \multirow{5}{3cm}{1.027}\\  
   & Moderate Extent & 3 & 21.4 & &\\ 
   & Large Extent & 3 & 21.4 & &\\ 
   & Very Large Extent & 7 & 50.0 & &\\ 
   & Total & 14 & 100 & &\\ \hline

   \multirow{5}{3cm}{Continuously aligning academic programs to vision 2030 and new constitution} & Moderate Extent & 3 & 21.4 & \multirow{5}{3cm}{3.36} & \multirow{5}{3cm}{0.929}\\  
   & Large Extent & 4 & 28.6 & & \\ 
   & Very Large Extent & 6 & 42.9 & &\\ 
   & None & 1 & 7.1 & &\\ 
   & Total & 14 & 100 & &\\ \hline

   \multirow{5}{3cm}{Continuously rolling out tech projects by students to the industry} & Less Extent & 1 & 7.1 & \multirow{5}{3cm}{2.86} & \multirow{5}{3cm}{0.949}\\  
   & Moderate Extent & 4 & 28.6 & & \\ 
   & Large Extent & 5 & 35.7 & &\\ 
   & Very Large Extent & 4 & 28.6 & &\\ 
   & Total & 14 & 100 & &\\ \hline

   \multirow{6}{3cm}{Regular reviews and renews its tech content objectives} & Less Extent & 2 & 14.3 & \multirow{6}{3cm}{2.86} & \multirow{6}{3cm}{1.027}\\  
   & Moderate Extent & 1 & 7.1 & & \\ 
   & Large Extent & 9 & 64.3 & &\\ 
   & Very Large Extent & 1 & 7.1 & &\\ 
   & None & 1 & 7.1 & &\\ 
   & Total & 14 & 100 & &\\ \hline

   \multirow{6}{3cm}{Continuous review of tech system} & Less Extent & 1 & 7.1 & \multirow{6}{3cm}{2.92} & \multirow{6}{3cm}{0.954}\\  
   & Moderate Extent & 2 & 14.3 & & \\ 
   & Large Extent & 8 & 57.1 & &\\ 
   & Very Large Extent & 1 & 7.1 & &\\ 
   & None & 1 & 7.1 & &\\ 
   & Total & 13 & 92.9 & &\\ \hline

   \multirow{5}{3cm}{Continuously involved in hosting tech events and seminars open to the public} & Moderate extent & 1 & 7.1 & \multirow{5}{3cm}{3.43} & \multirow{5}{3cm}{0.852}\\  
   & Large Extent & 8 & 57.1 & &\\ 
   & Very Large Extent & 3 & 21.4 & &\\ 
   & None & 2 & 14.3 & &\\ 
   & Total & 14 & 100 & &\\ \hline
    \end{tabular}
    \caption{ \textbf{Mean Responses on Innovation}}
\end{table}

The key objective of the study was to establish the innovation and performance of public universities in Nairobi. Before examining the influence, the study sought first to establish the extent to which the sampled public universities embraced various dimensions of innovations. These included product innovation, organizational and process innovation. The respondents were required to indicate the extent to which state of innovations applied to their respective universities on a Likert scale of 1-5 where this was based on the scale; 0.1-1.0- Less extent, 1.1-2.0- moderate extent, 2.1-3.0- Large extent, 3.1-4.0 – very large extent and 4.1-5.0- None. Seventeen dimensions of innovation were considered as represented in the table above.

It shows that the responses obtained on the innovation in universities greatly supported that the universities continuously engage in branding and marketing activities. This had the highest mean of 3.43 which was Continuously involved in hosting tech events and seminars open to the public meaning that the majority of the university do so to a very large extent. This had a standard deviation of 0.852 meaning that if the study was carried out on the entire population rather than on the sample, the results obtained would be slightly different. The lowest mean obtained was 2.86 which was Regularly reviewed and renewed its tech content objectives. This had a standard deviation of 1.027 which also indicated that there could be a slight difference on the mean response obtained if the study was carried from the entire population. Generally, the results in the table indicate that the responses obtained on all the mentioned strategies were largely accepted and supported by the means obtained as all the means lie between 2.1-4.0 which implies a large extent of agreement.	

\section{\textbf{Innovation and Performance of the Public Universities}}
The objective of the study was to analyse the influence of innovations on the performance of the capital of Kenya public universities. This section presents the findings of the study on the influence of innovation on the performance of the chosen universities within the study.

Through hierarchic multiple correlation at ninety fifth confidence the character of the innovation result (positive or negative) on every of the structure performance indicators resolve. The outputs for the analysis were multiple R, R2, F test, among different outputs for the multiple result of the innovation on every of the performance indicators. The regression outputs for the freelance result of the innovation on the structure performance indicators area unit the standardized coefficients, beta weights and t take a look at among others. The t take a look at assesses the importance of the experimental variable on the variable quantity. The multiple R shows the strength of the connection between every of the performance indicators and also the innovation indicator. R2 is that the proportion of variance within the variable quantity explained severally or together by the independents variables. The F take a look at is employed to gauge the importance of the regression model as an entire.

The multivariate analysis results for every of the innovation indicators and also the structure performance indicators area unit given and mentioned below. The analysis assess the result of the joint innovation indicators additionally because the freelance result of the innovation indicators on resource generation.

\subsection{\textbf{Innovation and Resource Generation}}
To establish the influence of the innovation on the resource generation of the colleges elect, a multiple correlation analysis was undertaken. The indices for the market performance were calculated from the varied responses from the four resource generation indicators from the Likert scale form. the indications of resource generation were fund levels, costs/cost saving, ICT facilities, physical facilities and equipment’s, performance appraisals for the workers, worker satisfaction and client satisfaction. The joint impact of the innovation indicators on the resource generation is bestowed below:

\begin{table}[ht]
    \centering
\begin{tabular}{|l|l|l|l|l|}
\hline
   \textbf{N} & \textbf{R} & \textbf{R2} & \textbf{F} & \textbf{Sig.}\\ \hline
   5 & 0.81 & 0.88 & 45.32 & 0.022\\ \hline   
\end{tabular}
    \caption{\textbf{Joint effect of Innovations indicators on Resource Generation}}
\end{table}

\begin{enumerate}
    \item Dependent variable: Resource generation
    \item Predictor variable: Organization Innovation, Process Innovation, Product innovation.
\end{enumerate}

The results show that there's sturdy positive relationship between combined innovation indicators and resource generation of the general public universities (R = 0.81). The analysis reveals that eighty-six of the resource generation may be accounted for by the innovation (R2 = 0.88). The results any shows that the check of confidence (p value) is a smaller amount that the check level of 0.05 ($p < 0.05$). this implies that the study results square measure statistically important thence may be relied on to elucidate the resource generation of the general public universities.
Independent indicators of innovation were regressed to determine their result on resource generation. The results for the gradable multiple regressions for the freelance result of innovation on resource generation are shown below:

\begin{table}[ht]
    \centering
\begin{tabular}{|l|l|l|l|l|l|}
   \hline
   \textbf{Model} & \multicolumn{2}{|l|}{\textbf{Unstandardized coefficients}} & \multicolumn{3}{|l|}{\textbf{Standardized coefficients}}\\ \hline
   & \textbf{B} & \textbf{Std. Error} & \textbf{Beta} & \textbf{t} & \textbf{Sig.}\\ \hline
   \textbf{(Constant)} & 3.589 & .475 & 1.122 & 4.154 & .055\\ \hline
   Budgetary Level & -.455 & .113 & -1.122 & -3.486 & .065\\ \hline
   Cost Saving & -.0697 & .134 & -.673 & -.673 & .564\\ \hline
   ICT Facilities & .0896 & .105 & .207 & .575 & .255\\ \hline
   Physical Facilities and Equipment & -.505 & .123 & -.354 & -1.867 & .119\\ \hline
   Performance appraisals for the staff & .043 & .113 & .127 & .468 & .001\\ \hline
   Employee Satisfaction & .180 & .095 & .652 & 1.835 & .205\\ \hline
   Customer Satisfaction & -.450 & .111 & -.167 & -.576 & .035\\ \hline
\end{tabular}
    \caption{\textbf{Independent Effect of Innovations on resource generation}}
\end{table}

The results show that there's positive result between the innovation indicators of ICT facilities model ($\beta = 0.207$), performance appraisals for the employees ($\beta = 0.127$) and worker satisfaction ($\beta = 0.652$) additionally were found to own a positive result. Negative result is recorded for the remaining innovation indicators. Physical facilities and equipment’s ($\beta = -.354$, client satisfaction ($\beta = -0.167$), price saving ($\beta = -0.673$), and monetary fund level at ($\beta = -1.122$) registered negative result. The study reports statistically not vital results for all the freelance innovation indicators. The analysis any reveals that resource generation will increase by 3.597 variances once innovation will increase by one purpose once alternative variables area unit unbroken constant.

The freelance result of the innovation indicators on resource generation of the general public universities generates a regression model below. The variables within the model area unit given in chapter 3 below the info analysis sub section.

$Y1 = a1+ \beta1X1$ + $\beta2X2$ + $\beta3X3$+ $\beta4X4$ + $\beta5X5$ + $\beta6X6$ + $\beta7X7$

$Y1= 3.589 -.455X1 -0.0697X2 + 0.0896X3 -0.505X4 +0.043X5 + 0.180X6 -.450X7$

The multivariate analysis results for the innovation and resource generation indicate that the multiple indicators of the innovation have a major important result on the resource generation however severally the result isn't statistically significant. this suggests that innovation will solely be relied upon to buffer generation of resources once it's pursued as an entire as critical individual implementation of the one indicator.

\section{\textbf{Discussions}}
The researcher was able to capture 60\% response rate as he was only able to sample 3 universities out of the total 5 universities targeted in the study. From the study, it was found that most of the public universities in Nairobi have been in existence for about 1 and 5 years, followed by those that had been in existence for over 21 years. It was also established that the longer the university had been in existence, the larger the number of students it contained, since majority of the university had been in existence for 1-5 years, this group had a population of 10,000 and less. The group of universities that had been in existence for over 21 years had a student population of over 25000. There was also a relationship between the duration of existence and the expansion of the university. Universities with over 21 years of existence were found to have 7 and more campuses/ branches while those that had had a short period of existence had less than 7 with some even having none.

The researcher also established from the study that Nairobi public universities continuously introduced and implemented innovation practices such as introducing new programs, rolling out tech projects by students to the industry, aligning its academic programs to vision 2030 and the new constitution.

